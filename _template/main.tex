\documentclass[titlepage,landscape,a4paper,10pt]{article}
\usepackage{listings, color, fontspec, minted, setspace, titlesec, fancyhdr, dingbat, mdframed, multicol}
\usepackage{graphicx, amssymb, amsmath, textcomp, booktabs}
\usepackage[Chinese]{ucharclasses}
\usepackage[left=1.5cm, right=0.7cm, top=1.7cm, bottom=0.0cm]{geometry}

%configure the top corners
\pagestyle{fancy}
\setlength{\headsep}{0.1cm}
\rhead{Page \thepage}
\lhead{北京交通大学 Beijing JiaoTong University}

%configure space between the two columns
\setlength{\columnsep}{30pt}

%configure fonts
\setmonofont{Isotype}[Scale=0.8]
\newfontfamily\substitutefont{SimHei}[Scale=0.8]
\setTransitionsForChinese{\begingroup\substitutefont}{\endgroup}

%configure minted to display codes 
\definecolor{Gray}{rgb}{0.9,0.9,0.9}

%remove leading numbers in table of contents
\setcounter{secnumdepth}{0}

%configure section style
%\titleformat{\section}
%    {\normalfont\normalsize}    % The style of the section title
%    {}                    % a prefix
%    {0pt}                % How much space exists between the prefix and the title
%    {\quad}                % How the section is represented
\titleformat{\section}{\large}{}{0pt}{}
\titlespacing{\section}{0pt}{0pt}{0pt}

%enable section to start new page automatically
%\let\stdsection\section
%\renewcommand\section{\penalty-100\vfilneg\stdsection}

%\renewcommand\theFancyVerbLine{\arabic{FancyVerbLine}}
\renewcommand{\theFancyVerbLine}{\sffamily \textcolor[rgb]{0.5,0.5,0.5}{\scriptsize {\arabic{FancyVerbLine}}}}

\setminted[cpp]{
    style=xcode,
    mathescape,
    linenos,
    autogobble,
    baselinestretch=0.9,
    tabsize=2,
    fontsize=\normalsize,
    %bgcolor=Gray,
    frame=single,
    framesep=1mm,
    framerule=0.3pt,
    numbersep=1mm,
    breaklines=true,
    breaksymbolsepleft=2pt,
    %breaksymbolleft=\raisebox{0.8ex}{ \small\reflectbox{\carriagereturn}}, %not moe!
    %breaksymbolright=\small\carriagereturn,
    breakbytoken=false,
}
\setminted[java]{
    style=xcode,
    mathescape,
    linenos,
    autogobble,
    baselinestretch=1.0,
    tabsize=2,
    %bgcolor=Gray,
    frame=single,
    framesep=1mm,
    framerule=0.3pt,
    numbersep=1mm,
    breaklines=true,
    breaksymbolsepleft=2pt,
    %breaksymbolleft=\raisebox{0.8ex}{ \small\reflectbox{\carriagereturn}}, %not moe!
    %breaksymbolright=\small\carriagereturn,
    breakbytoken=false,
}
\setminted[text]{
    style=xcode,
    mathescape,
    linenos,
    autogobble,
    baselinestretch=1.0,
    tabsize=2,
    %bgcolor=Gray,
    frame=single,
    framesep=1mm,
    framerule=0.3pt,
    numbersep=1mm,
    breaklines=true,
    breaksymbolsepleft=2pt,
    %breaksymbolleft=\raisebox{0.8ex}{ \small\reflectbox{\carriagereturn}}, %not moe!
    %breaksymbolright=\small\carriagereturn,
    breakbytoken=false,
}

%configure titles
\title{\LARGE{Never Say Never} \\[2ex] \Large{Standard Code Library} }
\date{\today}

%THE SCL BEGINS
\begin{document}
\maketitle

\begin{multicols*}{2}

    \begin{spacing}{0}
        \tableofcontents
    \end{spacing}
\end{multicols*}

\begin{multicols}{2}

\newpage
\begin{spacing}{0.8}

\section{基础}

\subsection{头文件}
\inputminted{cpp}{Basic/headers.cpp}

\subsection{二进制函数 枚举组合数}
\inputminted{cpp}{Basic/binary.cpp}

\subsection{矩阵乘法}
\inputminted{cpp}{Basic/Matrix.cpp}

\section{IO}

\subsection{Cpp 快速读入}
\inputminted{cpp}{IO/fastio.cpp}

\subsection{Java 模板}
\inputminted{java}{IO/Main.java}

\section{数据结构}

\subsection{BIT}
\inputminted{cpp}{DataStructure/BIT.cpp}

\subsection{LCA}
\inputminted{cpp}{DataStructure/LCA.cpp}

\subsection{RMQ}
\inputminted{cpp}{DataStructure/RMQ.cpp}

\subsection{Trie}
\inputminted{cpp}{DataStructure/trie.cpp}

\subsection{二维线段树}
\inputminted{cpp}{DataStructure/二维线段树.cpp}

\subsection{主席树}
\inputminted{cpp}{DataStructure/主席树.cpp}

\subsection{树链剖分}
\inputminted{cpp}{DataStructure/树链剖分.cpp}

\section{图论}

\subsection{2-SAT}
\inputminted{cpp}{Graph/2-SAT.cpp}

\subsection{ISAP}
\inputminted{cpp}{Graph/ISAP.cpp}

\subsection{KM}
\inputminted{cpp}{Graph/KM.cpp}

\subsection{最小费用流}
\inputminted{cpp}{Graph/MinCostFlow.cpp}

\subsection{SAP}
\inputminted{cpp}{Graph/SAP.cpp}

\subsection{Dinic}
\inputminted{cpp}{Graph/dinic.cpp}

\subsection{Dijkstra 费用流}
\inputminted{cpp}{Graph/MinCostFlow.cpp}

\subsection{zkw 费用流}
\inputminted{cpp}{Graph/zkw费用流.cpp}

\subsection{欧拉回路}
\inputminted{cpp}{Graph/欧拉回路.cpp}

\subsection{二分图最大匹配 匈牙利算法}
\inputminted{cpp}{Graph/匈牙利算法.cpp}

\subsection{最小树形图}
\inputminted{cpp}{Graph/朱刘.cpp}

\subsection{哈密尔顿回路}
\inputminted{cpp}{Graph/哈密尔顿回路.cpp}

\subsection{增广路费用流}
\inputminted{cpp}{Graph/增广路费用流.cpp}

\subsection{无向图最小割}
\inputminted{cpp}{Graph/无向图最小割.cpp}

\subsection{一般图最大匹配 带花树}
\inputminted{cpp}{Graph/带花树.cpp}

\subsection{生成树计数}
\inputminted{cpp}{Graph/生成树计数.cpp}

\section{字符串}

\subsection{Hash}
\inputminted{cpp}{Strings/BKDRHash.cpp}

\subsection{KMP}
\inputminted{cpp}{Strings/KMP.cpp}

\subsection{EXKMP}
\inputminted{cpp}{Strings/EXKMP.cpp}

\subsection{SA}
\inputminted{cpp}{Strings/SA.cpp}

\subsection{Manacher 最长回文串}
\inputminted{cpp}{Strings/Manacher.cpp}

\subsection{最小表示法}
\inputminted{cpp}{Strings/最小表示法.cpp}

\section{数学}

\subsection{高斯消元}
\inputminted{cpp}{Math/高斯消元.cpp}

\section{数论}

\subsection{勒让德定理}
\inputminted{cpp}{NumberTheory/Legendre.cpp}

\subsection{欧拉函数}
\inputminted{cpp}{NumberTheory/欧拉函数.cpp}

\subsection{线性逆元}

\section{常用结论}
\subsection{vimrc}
syntax on \\
set nu \\
set cindent \\
set shiftwidth=4 \\
set tabstop=4 \\
\subsection{线性逆元}
$ inv(i) = (M - \frac{M}{i}) \cdot inv(M \mod i) \mod M $ \\
适用于$ M $是质数且$ M > n $的情况 , 能够$ O(n) $时间求出$ [1, n] $对模 MOD 的逆
\subsection{康托展开}
\paragraph{}
$ X = a_n  n! + a_{n-1}  (n - 1)! + ... + a_2 \cdot 2! + a1 \cdot 1! $ \\
$ a_i $ 表示 i 开头的后缀中逆序对的个数
\subsection{高阶等差数列}
    \paragraph{}
    \begin{eqnarray}
        && \sum n = \frac{1}{2}n(n+1)\\
        && \sum n^2 = \frac{1}{6}n(n+1)(2n+1) \\
        && \sum n^3 = \left(\sum n\right)^2 \\
        && \sum n^4 = \left(\sum n^2\right)\frac{1}{5}(3n^2+3n-1)\\
        && \sum n^5 = \left(\sum n\right)^2\frac{1}{3}(2n^2+2n-1)\\
        && \sum n^6 = \left(\sum n^2\right)\frac{1}{7}(3n^4+6n^3-3n+1)\\
        && \sum n^7 = \left(\sum n\right)^2\frac{1}{6}(3n^4+6n^3-n^2-4n+2)\\
        && \sum n^8 = \left(\sum n^2\right)\frac{1}{15}(5n^6+15n^5+5n^4-15n^3-n^2+9n-3)
    \end{eqnarray}

\subsection{割建图}
对于图G=(V,E)中的一个点覆盖是一个集合S⊆V使得每一条边至少有一个端点在S中 . \\
        对于二分图 , \\
        最小路径覆盖 = | P | - 最大匹配数 . \\
        最小点覆盖数 = 最大匹配数 , \\
        最大独立顶点集 = 总顶点数 - 最大匹配数 , \\
\subsection{哈密尔顿判定}
对于$ n >= 3 $个点的图$ G $ , 如果对于任意$ u, v $ 都有  $ deg(u) + deg(v) >= n $ , 则$ G $一定是哈密尔顿图 .
\subsection{弦图}
设 $next(v)$ 表示 $N(v)$ 中最前的点 . 
令 $w*$ 表示所有满足 $A \in B$ 的 $w$ 中最后的一个点 , 
判断 $v \cup N(v)$ 是否为极大团 , 
只需判断是否存在一个 $w \in w*$, 
满足 $Next(w)=v$ 且 $|N(v)| + 1 \leq |N(w)|$ 即可 . 
\subsection{五边形数}
$
\prod_{n=1}^{\infty}{(1-x^{n})}=\sum_{n=0}^{\infty}{(-1)^{n}(1-x^{2n+1})x^{n(3n+1)/2}} \\
f(n) = \frac{n(3n - 1)}{2}
$
\subsection{重心}
半径为 $r$ , 圆心角为 $\theta$ 的扇形重心与圆心的距离为 $\frac{4r\sin(\theta/2)}{3\theta}$ \\
半径为 $r$ , 圆心角为 $\theta$ 的圆弧重心与圆心的距离为 $\frac{4r\sin^3(\theta/2)}{3(\theta-\sin(\theta))}$ \\
\subsection{第二类 Bernoulli number}
\begin{align*}
    B_m &= 1 - \sum_{k=0}^{m-1}{\binom{m}{k}\frac{B_{k}}{m-k+1}} \\
    S_m(n) &= \sum_{k=1}^{n}{k^{m}} = \frac{1}{m+1}\sum_{k=0}^{m}{\binom{m+1}{k}B_{k}n^{m+1-k}}
\end{align*}
\subsection{Stirling 数}
第一类 :n 个元素的项目分作 k 个环排列的方法数目\\
\begin{align*}
    s(n, k) &= (-1)^{n+k}|s(n, k)| \\
    |s(n, 0)| &=0\\ 
    |s(1, 1)| &=1 \\
    |s(n, k)| &=|s(n-1, k-1)|+(n-1)*|s(n-1, k)|
\end{align*}
第二类 :n 个元素的集定义 k 个等价类的方法数\\
\begin{align*}
    S(n,1)&=S(n,n)=1\\
    S(n,k)&=S(n-1,k-1)+k*S(n-1,k)
\end{align*}

\subsection{数据范围}


\subsection{三角公式}

\begin{footnotesize}
\noindent
\mbox{\vbox to 11pt{  \hbox{$
\sin(a \pm b) = \sin a \cos b \pm \cos a \sin b
$}  }}
\
\mbox{\vbox to 11pt{  \hbox{$
\cos(a \pm b) = \cos a \cos b \mp \sin a \sin b
$}  }}
\\
\mbox{\vbox to 11pt{  \hbox{$
\tan(a \pm b) = \frac{\tan(a)\pm\tan(b)}{1 \mp \tan(a)\tan(b)}
$}  }}
\
\mbox{\vbox to 11pt{  \hbox{$
\tan(a) \pm \tan(b) = \frac{\sin(a \pm b)}{\cos(a)\cos(b)}
$}  }}
\\
\mbox{\vbox to 11pt{  \hbox{$
\sin(a) + \sin(b) = 2\sin(\frac{a + b}{2})\cos(\frac{a - b}{2})
$}  }}
\
\mbox{\vbox to 11pt{  \hbox{$
\sin(a) - \sin(b) = 2\cos(\frac{a + b}{2})\sin(\frac{a - b}{2})
$}  }}
\\
\mbox{\vbox to 11pt{  \hbox{$
\cos(a) + \cos(b) = 2\cos(\frac{a + b}{2})\cos(\frac{a - b}{2})
$}  }}
\
\mbox{\vbox to 11pt{  \hbox{$
\cos(a) - \cos(b) = -2\sin(\frac{a + b}{2})\sin(\frac{a - b}{2})
$}  }}
\\
\mbox{\vbox to 11pt{  \hbox{$
\sin(na) = n\cos^{n-1}a\sin a - \binom{n}{3}\cos^{n-3}a \sin^3a + \binom{n}{5}\cos^{n-5}a\sin^5a - \dots
$}  }}
\\
\mbox{\vbox to 11pt{  \hbox{$
\cos(na) = \cos^{n}a - \binom{n}{2}\cos^{n-2}a \sin^2a + \binom{n}{4}\cos^{n-4}a\sin^4a - \dots
$}  }}

\end{footnotesize}
\subsection{积分表}
\begin{footnotesize}
\noindent
\mbox{\vbox to 11pt{  \hbox{$
\int \frac{1}{1+x^2}dx = \tan^{-1}x
$}  }}
\
\mbox{\vbox to 11pt{  \hbox{$
\int \frac{1}{a^2+x^2}dx = \frac{1}{a}\tan^{-1}\frac{x}{a}
$}  }}
\\
\mbox{\vbox to 11pt{  \hbox{$
\int \frac{x}{a^2+x^2}dx = \frac{1}{2}\ln|a^2+x^2|
$}  }}
\
\mbox{\vbox to 11pt{  \hbox{$
\int \frac{x^2}{a^2+x^2}dx = x-a\tan^{-1}\frac{x}{a}
$}  }}
\\
\mbox{\vbox to 11pt{  \hbox{$
\int\sqrt{x^2 \pm a^2} dx  = \frac{1}{2}x\sqrt{x^2\pm a^2} 
%\nonumber \\ 
\pm\frac{1}{2}a^2 \ln \left | x + \sqrt{x^2\pm a^2} \right | 
$}  }}
\\
\mbox{\vbox to 11pt{  \hbox{$
\int  \sqrt{a^2 - x^2} dx  = \frac{1}{2} x \sqrt{a^2-x^2} 
%\nonumber \\  
+\frac{1}{2}a^2\tan^{-1}\frac{x}{\sqrt{a^2-x^2}}
$}  }}
\\
\mbox{\vbox to 11pt{  \hbox{$
\int \frac{x^2}{\sqrt{x^2 \pm a^2}} dx  = \frac{1}{2}x\sqrt{x^2 \pm a^2}
%\nonumber \\  
\mp \frac{1}{2}a^2 \ln \left| x + \sqrt{x^2\pm a^2} \right | 
$}  }}
\\
\mbox{\vbox to 11pt{  \hbox{$
\int \frac{1}{\sqrt{x^2 \pm a^2}} dx = \ln \left | x + \sqrt{x^2 \pm a^2} \right | 
$}  }}
\\
\mbox{\vbox to 11pt{  \hbox{$
\int \frac{1}{\sqrt{a^2 - x^2}} dx = \sin^{-1}\frac{x}{a} 
$}  }}
\
\mbox{\vbox to 11pt{  \hbox{$
\int \frac{x}{\sqrt{x^2\pm a^2}}dx = \sqrt{x^2 \pm a^2} 
$}  }}
\
\mbox{\vbox to 11pt{  \hbox{$
\int \frac{x}{\sqrt{a^2-x^2}}dx = -\sqrt{a^2-x^2} 
$}  }}
\\
\mbox{\vbox to 11pt{  \hbox{$
\int  \sqrt{a x^2 + b x + c} dx = 
\frac{b+2ax}{4a}\sqrt{ax^2+bx+c}
\nonumber \\  
+
\frac{4ac-b^2}{8a^{3/2}}\ln \left| 2ax + b + 2\sqrt{a(ax^2+bx^+c)}\right |
$}  }}
\\
\mbox{\vbox to 11pt{  \hbox{$
\int x^n e^{ax}\hspace{1pt}\text{d}x = \dfrac{x^n e^{ax}}{a} - 
\dfrac{n}{a}\int x^{n-1}e^{ax}\hspace{1pt}\text{d}x
$}  }} 
\\
\mbox{\vbox to 11pt{  \hbox{$
\int \sin^2 ax dx = \frac{x}{2} - \frac{1} {4a} \sin{2ax}
$}  }}
\
\mbox{\vbox to 11pt{  \hbox{$
\int \sin^3 ax dx = -\frac{3 \cos ax}{4a} + \frac{\cos 3ax} {12a} 
$}  }}
\\
\mbox{\vbox to 11pt{  \hbox{$
\int \cos^2 ax dx = \frac{x}{2}+\frac{ \sin 2ax}{4a} 
$}  }}
\
\mbox{\vbox to 11pt{  \hbox{$
\int \cos^3 ax dx = \frac{3 \sin ax}{4a}+\frac{ \sin 3ax}{12a} 
$}  }}
\\
\mbox{\vbox to 11pt{  \hbox{$
\int \tan ax dx = -\frac{1}{a} \ln \cos ax 
$}  }}
\
\mbox{\vbox to 11pt{  \hbox{$
\int \tan^2 ax dx = -x + \frac{1}{a} \tan ax 
$}  }}
\\
\mbox{\vbox to 11pt{  \hbox{$
\int x \cos ax dx = \frac{1}{a^2} \cos ax + \frac{x}{a} \sin ax 
$}  }}
\
\mbox{\vbox to 11pt{  \hbox{$
\int x^2 \cos ax dx = \frac{2 x \cos ax }{a^2} + \frac{ a^2 x^2 - 2  }{a^3} \sin ax 
$}  }}
\\
\mbox{\vbox to 11pt{  \hbox{$
\int x \sin ax dx = -\frac{x \cos ax}{a} + \frac{\sin ax}{a^2} 
$}  }}
\
\mbox{\vbox to 11pt{  \hbox{$
\int x^2 \sin ax dx =\frac{2-a^2x^2}{a^3}\cos ax +\frac{ 2 x \sin ax}{a^2} 
$}  }}
\end{footnotesize}


\end{spacing}
\end{multicols}

\end{document}
