\subsection{最小树形图}
	\paragraph{}
		最小树形图, 就是给有向带权图中指定一个特殊的点root, 求一棵以root为根的有向生成树T, 并且T中所有边的总权值最小。
	\paragraph{}	
	最小树形图(根固定), $ O(VE) $。
    求一个有向图的最小生成树, 如果根不固定,添加一个根节点与所有点连无穷大的边(或者总边权+1), 如果求出比2*MOD大或者返回值为-1, 则不连通; 求根, 则求和虚拟根相连的结点。
	根据pre的信息能构造出这棵树, 注意结点必须从$ [0, n) $, 因为要考虑重新标号建图的统一, mytype根据实际情况确定
\begin{lstlisting}[language=C++]
const int Maxn = 1000;
const double MOD = 1e9;
struct obj
{
    int u, v;
    double w;
}e[Maxn * Maxn];
int n, m;

typedef double mytype;
int visit[Maxn], pre[Maxn], belong[Maxn], ROOT;
mytype inv[Maxn];
mytype dirtree(int n, int m, int root) {
    mytype sum = 0;
    int i, j, k, u, v;
    while (true) {
        for (i = 0; i < n; i++) {
            inv[i] = MOD;
            pre[i] = -1;
            belong[i] = -1;
            visit[i] = -1;
        }
        inv[root] = 0;
        for (i = 0; i < m; i++) { // `除原点外,找每个点的最小入边`
            u = e[i].u; v = e[i].v;
            if (u != v) {
                if (e[i].w < inv[v]) {
                    inv[v] = e[i].w;
                    pre[v] = u;
                    if(u == root) ROOT = i; // `记录根所在的边` 
                                            // `输出根时利用ROOT-m计算是原图哪个结点`
                }
            }
        }
        for (i = 0; i < n; i++) {

            if (inv[i] == MOD) return -1;
        }
        int num = 0;
        for (i = 0; i < n; i++) { // `找圈,收缩圈`
            if (visit[i] == -1) {
                j = i;
                for(j = i; j != -1 && visit[j] == -1 && j != root; j = pre[j]) {
                    visit[j] = i;
                }
                if (j != -1 && visit[j] == i) {
                    for (k = pre[j]; k != j; k = pre[k]) {
                        belong[k] = num;
                    }
                    belong[j] = num++ ;
                }
            }
            sum += inv[i];
        }
        if (num == 0) return sum;
        for (i = 0; i < n; i++){
            if (belong[i] == -1) {
                belong[i] = num ++ ;
            }
        }
        for (i = 0; i < m; i++) { // `重新构图`
            e[i].w = e[i].w - inv[e[i].v];
            e[i].v = belong[e[i].v];
            e[i].u = belong[e[i].u];
        }
        n = num;
        root = belong[root];
    }
}
\end{lstlisting}