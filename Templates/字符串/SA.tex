\subsection{SA}
\paragraph{}
倍增算法, r为待匹配数组, n为总长度+1, m为字符范围, num保存字符串
使用时注意num[]有效位为[0, n), 但是需要将num[n] = 0, 另外, 对于模板的处理将空串也处理了, 作为rank最小的串, 因此有效串为[0, n]共n-1个, 在调用da()函数时, 需要调用da(num, n + 1, m) 对于 sa[], rank[], height[] 数组都将空串考虑在内, 作为rank最小的后缀
注意rank, height范围从[0, n]
\begin{lstlisting}[language=C++]
	
namespace SA
{
    int len;
    int num[N];
    int sa[N], rank[N], height[N]; // sa[1~n]value(0~n-1); rank[0..n-1]value(1..n); height[2..n]
    int wa[N], wb[N], wv[N], wd[N];

    int cmp(int *r, int a, int b, int x) {
        return r[a] == r[b] && r[a + x] == r[b + x];
    }

    void da(int *r, int n, int m) {
        int i, j, k, p, *x = wa, *y = wb, *t;
        for(i = 0; i < m; i++) wd[i] = 0;
     for(i = 0; i < n; i++) wd[x[i] = r[i]]++;
        for(i = 1; i < m; i++) wd[i] += wd[i - 1];
        for(i = n - 1; i >= 0; i--) sa[--wd[x[i]]] = i;
        for(j = 1, p = 1; p < n; j <<= 1, m = p) {
            for(p = 0, i = n - j; i < n; i++) y[p++] = i;
            for(i = 0; i < n; i++) if(sa[i] >= j) y[p++] = sa[i] - j;
            for(i = 0; i < n; i++) wv[i] = x[y[i]];
            for(i = 0; i < m; i++) wd[i] = 0;
            for(i = 0; i < n; i++) wd[wv[i]]++;
            for(i = 1; i < m; i++) wd[i] += wd[i - 1];
            for(i = n - 1; i >= 0; i--) sa[--wd[wv[i]]] = y[i];
            for(t = x, x = y, y = t, p = 1, x[sa[0]] = 0, i = 1; i < n; i++) {
                x[sa[i]] = cmp(y, sa[i - 1], sa[i], j) ? p - 1 : p++;
            }
        }

        for(i = 0, k = 0; i < n; i++) rank[sa[i]] = i;
        for(i = 0; i < n - 1; height[rank[i++]] = k) {
            for(k ? k-- : 0, j = sa[rank[i] - 1]; r[i + k] == r[j + k]; k++);
        }
    }
}
	\end{lstlisting}